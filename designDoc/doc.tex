\documentclass[10pt,draftclsnofoot,onecolumn]{IEEEtran}
\newcommand{\namesigdate}[2][7cm]{%
\begin{minipage}{#1}
    #2 \vspace{1.0cm}\hrule\smallskip
    \small \textit{Signature}
    \vspace{1.0cm}\hrule\smallskip
    \small \textit{Date}
\end{minipage}
}
\usepackage{graphicx}
\usepackage{textcomp}
\usepackage{url}
\usepackage{cite}
\graphicspath{ {} }
\begin{document}
\pagenumbering{gobble}
\title{One Clock To Rule Them All - Product Design}
\author{Tristan Hari, Tasman Thenell, and Scott Metzsch}
\maketitle
\begin{abstract}
The purpose of this document is to discuss the design of our project.
Each requirement for the project needs to be designed around and those design decisions are dicussed in this document.
The design choices made at this point serve as a starting point for implementation of this project and will be updated over time.
The final iteration of this document with include all design requirements, specifications, and design choices.
\end{abstract}

\newpage

\section{Scope}
This product will an improved experience for users seeking an interesting and unconventional timekeeping device.
It will provide a fashionable and affordable alternative to a class of product that is usually very expensive.
In addition to these goals, the product will also provide an opensource platform for users with the skills to modify the programming of the device.

\section{Purpose and Background}
The intention of this document is to breakdown the details of this project, including the steps we will take to complete it and the tools/methods we will use to effectively develop our product.
The metrics that we will use to measure product completion will also be included in this document, and will be referenced as we continue to develop throughout the year.
The background of the product arises from the personal interest of our client, Victor Hsu.
While there are similar products already on the market, QWLock by Ziegert and Funk for example, these products suffer from high prices and low avaliability.
Thses items traditional occupy a designer marker niche which doesn't allow most people to experience one.
In addition to these problems, it is also difficult to track down a model that has english lettering, as Ziegert is based in Germany.

We are not the first group of people to be inspired by the unique asthetic and function of the word clock.
There are plenty of DIY projects online that also attempt to solve the problems of price and avaliability.
While these projects have been a valuable source of insight for the project, they are often very complicated and considered outside the comfort level of our expected users.
The intention of this product is to improve upon the situation by making the creation of a word clock less expensive, simpler, or both.

\section{Intended Audience}
The primary intended audience for this document is our client, Victor Hsu.
This document is intened to provide an approachable explanation to the project design to any inquiring individual.
That being said, it also includes enough detail to serve as a resource for individuals with low level background skills in electrical engineering or computer science who are intending on modifying the system.

\section{Definitions}
\begin{enumerate}[]
  \item Time - The abstract concept that our machine is keeping track of and displaying to the user
  \item Real Time Clock - A real time clock module is the hardware component that tracks the passage
  of time regardless of the state of the rest of the system.
  \item RTC - Acronym for Real Time Clock module.
  \item Microcontroller - The brains of the clock. This module interfaces between the hardware
  components, the RTC, LEDs and buttons, and processes the logic for all clock functionality.
  This includes tasks like updating the display and processing user input from the buttons.
  \item Programmable LED - A Light Emitting Diode that has variable levels of brightness and
  color output. Each module is individually programmable in terms of brightness and color.
  \item LED - Acronym for Light Emitting Diode. Within this project, the term LED will always
  be used to refer to a programmable light emitting diode.
  \item Display - The LED lit letters of the clock face in a grid arrangement are collectively
  called the display. This term will be used to refer to the visual
  \item Software Library - The set of control software running on the microcontroller which drives all display output.
  \item User Interface - A set of buttons by which the user controls and interacts with the clock.
  \item BCD - Acronym for Binary Coded Decimal.
  \item Binary Coded Decimal - A method for encoding decimal numbers and information via binary.
  \item USB - Acronym for Universal Serial Bus.
  \item Universal Serial Bus - A communications medium by which the device is programmed and  powered.
\end{enumerate}

\newpage

\section{Design Context}
When designing software for our clock we will be influenced heavily by the hardware components that we were given.
Our code will have sections written specifically for use with these intended components since the microcontroller, RTC, and LED strips all have specific ways they want data transfered.
These hardware requirements will have to be taken into account if someone is using different components to create our word clock.

\section{Design Concerns}

For our clock the main concerns are to be able to read the clock and make sure that it is accurate and does not lose time.
We will also be concerned the clocks ability to maintain power and charge effectively.
This will of course affect the main concerns listed, but was worth mentioning on its own as it is it's own unit of the product separate from the pieces that will control readability and timekeeping.

\subsection{Functionality}
The time will be set through a physical interface of 4 buttons, the clock will be able to recover from any set of presses from these buttons.
Through these buttons you will be able to set a visual alarm and switch the display mode to BCD.
Also the design of the face of the clock should be able to be read from 10 feet away.

\section{Design Viewpoints}

We have selected three primary viewpoints from which to break down the design behind the requirements for the project. 
The primary viewpoints are algorithmic, composition, and interface.
The algorithmic viewpoint was selected in order to discuss the accuracy and performance of the combination of the project hardware and our planned system software.
The composition viewpoint provides a space to describe the components of the project and how they interact, while the interface viewpoint describes user centric design components.


\subsection{Algorithm}
This section addresses some of our client's requirements about clock performance, BCD display mode, and time loss per day.

\subsubsection{Clock Performance}
Design Concern: This addresses the design concern of being accurate and is primarily for the users.
The clock will need to be accurate or it is of zero worth as a clock.

Analytical Methods: Simply comparing the clocks ability to remain accurate to another atomic based clock.
The time we have to exaustively test this is limited and we will only at most be running 48 hour trials.

Rationale: We decided that this viewpoint was important because it directly impacts the users experience.
Our client Victor Hsu would also consider this an important aspect as a measure of success in development of the clock.
In terms of the project Clock Performance and algorithm is entirely what determines the worth of the product, thusly making this a top priority in terms of viewing the project.

\subsubsection{BCD Display Mode}

\subsubsection{Time Loss}

\subsection{Composition}
This section addresses some of our client's requirements about time loss per day, display modes, and clock controls.

\subsubsection{Time Loss}
Design Concern: This addresses the design concern of the clock being accurate over time which is important to the users. The clock will need to be accurate and not lose the time when the clock is unplugged.  

Analytic Methods: Comparing the RTC time to the time of the microcontroller will be done every set amount of time to maintain accuracy and recover the time of the clock if there should be a power disconnect and the microcontroller loses the time. Testing will be done though unplugging the microcontroller and then replugging the microcontroller to verify the time is still being kept.

Rationale: This viewpoint is important to the design of the clock because it affects the easy of use and accuracy of the clock. Victor Hsu would also consider time loss as an important aspect in the development and functionality of the clock. In terms of time loss and composition, the RTC will be the main unit to keep time since there will not be any power interruptions with it's independent battery and can then communicate the time to the microcontroller with wires so that there will always be a device keeping the time even in the event of power loss to the microcontroller.

\subsubsection{Display Modes}
Design Concern: This addresses the design concern of the clock being able to display the clock in BCD mode and make the clock visible at 10 feet. 

Analytics Methods: The LEDs are the main display and will allow the user to read the time based on the backlit letters on the face of the clock. Testing will be done though a script that runs the LEDs through all colors and brightnesses as well as checking that each letter on the clock face is able to be controlled individually. 

Rationale: This viewpoint is important to the design of the clock because it affects the readability of the clock face and the letters that it is lighting. Controlling the brightness and color of the clock will help to meet Victor's requirement that the clock should be visible from 10 feet and can be also changed to displaying the time in BCD mode. These strips of LEDs will be connected  

\subsubsection{Clock Controls}
Design Concern: This addresses the design concern of the ability to control the clock and make changes to display settings and the time.

Analytics Methods: The buttons will be connected to the clock through a set of wires for each clock and will be tested through pressing each button and verifying that the microcontroller gets the input in terms of a press and a hold of the button. 

Rationale: This viewpoint is important to the design of the clock because it affects the easy of use when choosing different settings for the clock LEDs or setting the time. Victor would agree that having buttons that work and are accurate to their function is important to the design of the clock.  

\subsection{Interface}
This section addresses our client's requirements about clock controls and ability to read the clock.
The interface is broken down into two perspectives: users and developers.
The user centric interface consists of a set of buttons for menu control while the developer interface consists of direct interaction with the microcontroller.

\subsubsection{Clock Controls} 

The main design concern for clock controls was providing enough buttons for intuitive menu navigation without uneccesary complication.
A secondary design goal was to provide developers enough sources of input for any additional functionality.

To meet these design goals, a layout of four buttons on the side of the clock has been chosen. 
These buttons provide control over the planned interface scheme which is: a select button, a back button, and buttons for up and down scrolling.
An initial setup with only three buttons was considered but it did not make the interface simple enough from a user perspective.
The rational behind this was a desire to not have any button have more than one function in the core interface.
With the three button interface, that would have meant a contextual combination back and select button, or only scrolling in a single direction.

From the developer perspective, four buttons also won out as it provides additional flexibility.
While it is not known at this time if all of these buttons will be necessary, the developer perspective is secondary to the user one.
That being said, giving additional flexibility for the physical interface for any custom project was deemed to be mostly a positive.

\subsubsection{Clock Readability}

While being able to control the product is important, an even larger concern was the legibility of the clock face.
The primary purpose of a clock is to provide convienent access to the time of day.
With any creative twist on timekeeping comes the concern that the creative aspects will get in the way of the core functionality.

From the user perspective this design goal will be easy to meet.
Unlike other novel approaches to displaying time, our display doesn't require any additional thought or effort to aquire the time.
The whole idea behind a word clock is that reading the time is as simple as reading the clock face.
The clockface spells out the time and words and reads like a sentence.

While the display works very well from the user perspective for timekeeping, it comes with some large limitations for developers.
Unlike most other traditional displays, a letter based matrix is drastically lower resolution and does not use uniformly shaped pixels.
These concerns are secondary to user legibility but some choices have been made to help developers.
The current design calls for a grid which is fourteen by fourteen letters which is a larger resolution that most word clocks.
Providing a slightly higher resolution is the best we can do to help with the display of developer projects.

\subsubsection{Developer Interface}

One final interface design goal was to cater to developers or users that would like to tinker with the clock software.
We would like to facilitate access to the control software without the need for expensive software tools or hardware modification.

This design goal has been one of the easiest to meet due to the choice of microcontroller.
The choosen controller has a standard hardware interface via a microUSB port. 
In addition to this, it also comes with a web based integrated developer environment which provides software developer tools with the need for downloading or purchasing software.
These features of the microcontroller combined with making the default software open source should allow anyone who would like to tinker with the product or develop for it do so without much difficulty.

\end{document}
