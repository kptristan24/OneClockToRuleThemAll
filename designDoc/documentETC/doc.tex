\documentclass[10pt,draftclsnofoot,onecolumn]{IEEEtran}
\newcommand{\namesigdate}[2][7cm]{%
\begin{minipage}{#1}
    #2 \vspace{1.0cm}\hrule\smallskip
    \small \textit{Signature}
    \vspace{1.0cm}\hrule\smallskip
    \small \textit{Date}
\end{minipage}
}
\usepackage{graphicx}
\usepackage{textcomp}
\graphicspath{ {} }
\begin{document}
\pagenumbering{gobble}
\title{One Clock To Rule Them All}
\author{Tristan Hari, Tasman Thenell, and Scott Metzsch}
\maketitle
\begin{abstract}
Historically, the wristwatch has had one of two faces, analog or digital. These
formats have provided a solid mix between function and fashion, but there are
many other possible ways of visually representing the concept of time. One
creative idea worth exploring is timekeeping through words and letters. The
basis of our capstone project is to investigate making such a clock with a
microcontroller, real time clock module, and a set of LED’s. We will combine
this physical base with a software library of our creation to power a word clock.
\end{abstract}
\IEEEpeerreviewmaketitle

\newpage
\pagenumbering{arabic}

\title{Problem Statement}
\section{Problem Background}
Timekeeping isn\textquotesingle t an advanced concept at a fundamental level. Since the first
cavepeople noticed that they had to shade their eyes from the sun differently
at various ``times'' of the day, humans have studied time. Throughout history
humans have created many different methods for tracking time. , each more
ostentatious than the last, for keeping time. Classic examples of the lengths
cultures have gone to track time include Stonehenge and Big Ben. While such
methods of timekeeping aren\textquotesingle t necessary, we still seek even more
interesting ways of telling time. One such example of this is the word clock. A
word clock has a face made out of a grid of letters. This grid is layed out such that
there are enough letters to create all words required for spelling out any given time
of day. While clocks following this idea already exist, most are overly complicated,
look chintzy, or are prohibitively expensive.

\section{Problem Solution}
We would like to address the problem of this type of time keeping solution not
being readily avaliable by creating a similar product using
off the shelf parts and our own custom software. The current concept for
implementation is to use a microcontroller, real time clock (RTC), and LED strips
with individually programmable lights to power the face of the clock. For the
casing/shell we will investigate the feasibility of using 3D printing, laser cutting,
or woodworking to create structural components. In addition to the hardware, we
will develop a software library to run on the microcontroller that supports
various modes of displaying time as well as providing any additional functionality
we have time to develop.

\section{Metrics for Completion}
In order to demonstrate completion of this project, we will meet the following requirements:
\begin{itemize}
  \item Establish a connection between a real time clock module and a
  microcontroller for the purpose of controlling a custom clock face.
  \item Design a grid of letters such that there are enough of each letter for displaying
  all of the words necessary for spelling out any time of day.
  \item Support for an alternate output mode for displaying the time in binary coded
  decimal (BCD) mode.
  \item Take at least one source of input to facilitate letting the user set the
  time on the clock.
\end{itemize}

\section{Stretch/Additional Goals}
If time permits, our group will explore additional features that would either
make the clock more robust or easier \quad to use. The following stretch goals have
already been discussed:
\begin{itemize}
  \item Wireless or bluetooth methods for setting or changing the clock remotely,
  possibly via a mobile app.
  \item Support alternate physical configurations or letter grid arrangements.
  \item Different view modes that use the eleven by eleven grid of letters as an
  eleven by eleven pixel low resolution screen.
\end{itemize}

\newpage
\newpage

\noindent \namesigdate{Victor Hsu} \hfill \namesigdate{Tristan Hari} \par
\vspace{2cm}
\noindent \namesigdate{Tasman Thenell} \hfill \namesigdate{Scott Metzsch}
\end{document}
