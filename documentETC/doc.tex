\documentclass[10pt,draftclsnofoot,onecolumn]{IEEEtran}
\newcommand{\namesigdate}[2][5cm]{%
\begin{minipage}{#1}
    #2 \vspace{1.0cm}\hrule\smallskip
    \small \textit{Signature}
    \vspace{1.0cm}\hrule\smallskip
    \small \textit{Date}
\end{minipage}
}
\usepackage{graphicx}
\graphicspath{ {} }
\begin{document}
\pagenumbering{gobble}
\title{One Clock To Rule Them All}
\author{Tristan Hari, Tasman Thenell, and Scott Metzsch}
\maketitle
\begin{abstract}
Historically, the wristwatch has had one of two faces, analog or digital. These
formats have provided a solid mix between function and fashion, but there are
many other possible ways of visually representing the concept of time. One
creative idea worth exploring is timekeeping through words and letters. The
basis of our capstone project is to investigate making such a clock with a
microcontroller, real time clock module, and a set of LED’s. We will combine
this physical base with a software library of our creation to power a word clock.
\end{abstract}
\IEEEpeerreviewmaketitle

\newpage
\pagenumbering{arabic}

\title{Problem Statement}
\section{Problem Background}
Timekeeping isn’t an advanced concept at a fundamental level. Since the first
caveman noticed that he or she had to shade their eyes from the sun differently
at various “times” of the day, humans have studied time. Throughout history
humans have created many different physical representations, each more
ostentatious than the last, for keeping time. Classic examples of the lengths
cultures have gone to track time include Stonehenge and Big Ben. While such
methods of timekeeping aren’t always necessary, we still seek even more
interesting ways of telling time. One such example of this is the word clock. A
word clock has a face made out of grid of letters such that there are enough
letters to create words for spelling out any given time of day. While clocks
following this idea already exist, most are very rare, look chintzy, or are
prohibitively expensive for most people.

\section{Problem Solution}
We would like to address this problem by creating a similar product using
cheaper off the shelf parts some of our own custom software. The current idea
will be to use a Mikrobus controller with a Keyestudios clock module to provide
for the basic hardware components, as well as several LED strips to backlight
the letters. For the casing/shell we will investigate the feasibility of 3d
printing or woodworking. In addition to the hardware, we will develope a
software library to run on the microcontroller that supports the various modes
of displaying time as well as to provide any additional functionality we have
time to develop.

\section{Metrics for Completion}
In order to demonstrate completion for this project, we will meet the following requirements:
\begin{itemize}
  \item Establish a working connection between a real time clock module and a
  microcontroller for the purpose of controlling a custom clock face.
  \item Given a valid layout of letters in a visual format (i.e. an arrangement
  with enough letters to spell all the necessary combinations of words for
  timekeeping) the software to drive a word clock via our hardware module
  described above.
  \item Support an alternate output mode for displaying the time in binary coded
  decimal (BCD) mode.
  \item Take at least one source of input to facilitate letting the user set the
  time on the clock.
\end{itemize}

\section{Stretch/Additional Goals}
If time permits, our group will explore additional features that would either
make the clock more robust or easier to use. The following stretch goals have
already been discussed:
\begin{itemize}
  \item Wireless or bluetooth methods for setting or changing the clock remotely,
  possibly via a mobile app.
  \item Support alternate physical configurations or letter grid arrangements.
  \item Different view modes that use the eleven by eleven grid of letters as an
  eleven by eleven pixel low resolution screen.
\end{itemize}

\newpage
\newpage

\noindent \namesigdate{Victor Hsu} \hfill \namesigdate{Tristan Hari}
\noindent \namesigdate{Tasman Thenell} \hfill \namesigdate{Scott Metzsch}
\end{document}
