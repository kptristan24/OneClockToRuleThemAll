\documentclass[10pt,draftclsnofoot,onecolumn]{IEEEtran}
\usepackage{graphicx}
\graphicspath{ {} }
\begin{document}
\pagenumbering{gobble}
\title{Writing Assignment 4: Windows, FreeBSD, and Linux Memory Management}
\author{Tristan Hari}
\maketitle
\begin{abstract}
Historically, the wristwatch has had one of two faces, analog or digital. These
formats have provided a solid mix between function and fashion, but there are
many other possible ways of visually representing the concept of time. One
creative idea worth exploring is timekeeping through words and letters. The
basis of our capstone project is to investigate making such a clock with a
microcontroller, real time clock module, and a set of LED’s. We will combine
this physical base with a software library of our creation to power a word clock.
\end{abstract}
\IEEEpeerreviewmaketitle

\newpage
\pagenumbering{arabic}

\title{Problem Statement}
\section{Problem Background}
Timekeeping isn’t an advanced concept at a fundamental level. Since the first
caveman noticed that he or she had to shade their eyes from the sun differently
at various “times” of the day, humans have studied time. Throughout history
humans have created many different physical representations, each more
ostentatious than the last, for keeping time. Classic examples of the lengths
cultures have gone to track time include Stonehenge and Big Ben. While such
methods of timekeeping aren’t always necessary, we still seek even more
interesting ways of telling time. One such example of this is the word clock. A
word clock has a face made out of grid of letters such that there are enough
letters to create words for spelling out any given time of day. While clocks
following this idea already exist, most are very rare, look chintzy, or are
prohibitively expensive for most people.

\section{Problem Solution}
We would like to address this problem by creating a similar product using
cheaper off the shelf parts some of our own custom software. The current idea
will be to use a Mikrobus controller with a Keyestudios clock module to provide
for the basic hardware components, as well as several LED strips to backlight
the letters. For the casing/shell we will investigate the feasibility of 3d
printing or woodworking. In addition to the hardware, we will develope a
software library to run on the microcontroller that supports the various modes
of displaying time as well as to provide any additional functionality we have
time to develop.

\section{Metrics for Completion}
In order to demonstrate completion for this project, we will meet the following requirements:
\begin(itemize)
  \item Establish a working connection between a real time clock module and a
  microcontroller for the purpose of controlling a custom clock face.
  \item Given a valid layout of letters in a visual format (i.e. an arrangement
  with enough letters to spell all the necessary combinations of words for
  timekeeping) the software to drive a word clock via our hardware module
  described above.
  \item Support an alternate output mode for displaying the time in binary coded
  decimal (BCD) mode.
  \item Take at least one source of input to facilitate letting the user set the
  time on the clock.


\newpage


\section{Windows Implementation}

When it comes to memory management, windows has various different implementations
compared to Linux's, however, they Windows methods still operate to achieve
almost the same exact purpose. A good example is that the primary method in
Windows for MM is the usage of virtual memory. In Windows, each process on a 32
bit system gets its own virtual address space that enables addressing up to 4
gigabytes of memory. In 64 bit Windows, each process gets a virtual address space
of up to 8 TERAbytes. As one would almost expect, each thread of a process can
access the entire processes virtual address space, however, threads cannot access
memory that belongs to another process. This is the simple method of keeping
processes from corrupting eachother during runtime. Windows has several ways of
protecting its processes form corrupting each other aside from simply not allowing
them to do so, as this is not always a limitation you want to have on a process.
A common one is copy on write protection, which is an implementation that allows
multiple processes' virtual memory to share a physical page until one of the
processes actualyl writes to the page. This gives it the optimization of a lazy
evaluation, because it does not perform any additional operations until it is
necessary for the system to do so. Looking at thetwo diagrams below, you can
see an example of this. Originally both process 1 and 2 have their virtual
memory mapped to the same physical pages.



And then you can see as you can see, as Page 2 attempts to actually modify Page
B, all the contents of B are copied to a new page D, and the page table is updated
to prevent and funny business between the two processes.



\newpage
\section{FreeBSD Implementation}

In many of my papers I've discussed how Linux and FreeBSD are extremely similar
in how they operate most of the time, due their familial lineage. However this
case is to the extreme. I spent several days attempting to find a single
difference in how the two do memory management, and alas I could not for the life
of me. So with that being said, I will simply discuss the details of what I found
for the linux kernel, as they are entirely same but the Linux documentation
contains a lot more information of the same topic. I think this is purposefully
so that one isn't led to believe that the BSD docs will yield anything different.

Similarly to windows, memory management is primarily done using virtual memory.
Additionally, like windows, each thread is given its own virtual memory space
to operate within, that without special interactions set up, cannot access other
processes virtual memory. The hardwares own virtual memory functions keep memory
in use from being overwritten for virtual use. The Linux system provides several
additional functionalities to help manage memory better. The functionalities the
documentation covers are as follows:

Large Address Spaces: The kernel will make the virtual memory appear much larger
than the actual physical space that is available.

Individual virtual Memory for Each process: This is covered above, but still worth
mentioning here.

Memory Mapping: This is the ability to map files into a process address space.
In memory mapping, the contents of a files are linked directly into the virtual
memory of the regarded process.

Fair Allocation of Space: The system will attempt to fairly allocate available
physical among all running processes.

Shared Virtual Space Functionality: The system provides methods that allow
processes to share virtual memory in a safe manner, should that be something you
wish to include. Memory can also be shared using Inter Process Communication
methods through the kernel API for it.

When the processor is running a program, it reads the next instructions from
memory and decodes it. When it is decoding, it may need to fetch or store the
contents of a memory location. The processor will then proceed to executes the
instruction and move onto the next one. This method keeps the processor constantly
either accessing memory to fetch the next instruction, or to fetch and store
data. When it comes to using virtual memory for processes, these addresses are
not physical addresses, they are virtual addresses mapped to physical addresses.
The system is able to convert these addresses back and forth using the information
held within page tables. The figure below shows the virtual address spaces of
two processes, process X and  Y, as well as each of their page tables. These page
tables map each processes virtual memory into physical addresses in memory. This
shows that process X's virtual memory page number 0 is mapped directly to memory
in physical memory page number 1 and etc. This diagram shows how these virutal
spaces are created separately using same space physical memory.

The linux system also has some various methods of caching available for use to
enchance performance and allow some more advanced memory management techniques.

The main ones covered in the document are as follows:

Buffer Cache: This cache contains data that is used by block device drivers. The
buffers are of fixed sizes and contain blocks of information that are either
being read or written to. The block devices as a matter of fact are only accessed
using the buffer cache. If data can be found in the buffer cache, then the block
device does not need to be accessed for a read, and thusly increases performance
of the system.

Page Cache: The page cache is primarily used to speed up accessing images and
data on a disk. It caches the logical contents of a file one page at a time and
is accessed with the file and offset within the file. When a page is read into
memory from the disk, they are cached in the page cache.

Swap Cache: Only modified AKA dirty files are kept in the swap cache. While
these pages are not madified after being written to the swap cache, then next
time the page is swapped there is no need to write it to the swap cache as the
page is already contained within. In this case, the page will be discarded. This
saves the system a lot of costly swapping operations in a time where swapping is
being used heavily.



\newpage
\section{Summary}

In conclusion, when it comes to comparing Windows and Linux, there are many
differences as one would imagine. And in a repeating pattern among the other
topics we've covered in this class, while all three of these kernels differ in
the specifics of how they accomplish memory management, they all use the same
pieces to ensure that proper management occurs. For example, all the Operating
Systems use: Process unique virtual memory, access methods to share virtual
memory, and caching methods. Its fairly difficult to discern if any particular
kernel has an advantage over the other, however one thing that is certain is that
in regards to FreeBSD and Linux, I could not find a single difference. They
appear to operate almost exactly the same in terms of memory management, and
thusly they clearly have no advantage over eachother. As far as windows goes, it
uses more of an object oriented style for its memory manager, so foregoing the
overhead of learning the system and additional resources required, it is most
likely easier to understand at the end of the day. In general, there was a lot to
learn here about the way kernels handle memory management it honestly it is
much more complex than I originally imagined.
\end{document}
