\documentclass[10pt,draftclsnofoot,onecolumn]{IEEEtran}
\newcommand{\namesigdate}[2][7cm]{%
\begin{minipage}{#1}
    #2 \vspace{1.0cm}\hrule\smallskip
    \small \textit{Signature}
    \vspace{1.0cm}\hrule\smallskip
    \small \textit{Date}
\end{minipage}
}
\usepackage{graphicx}
\usepackage{textcomp}
\usepackage{url}
\usepackage{cite}
\graphicspath{ {} }
\begin{document}
\pagenumbering{gobble}
\title{One Clock To Rule Them All - Progress Report}
\author{Tristan Hari, Tasman Thenell, and Scott Metzsch}
\maketitle
\begin{abstract}
The purpose of this document is to discuss the progress of the One Clock to Rule Them All project and what our team has done during the fall term.
Included is an explanation of each document written during the design period and a summary of the progress made during each week of the term.
In addition, this process is reflected in retrospective which discusses challenges and difficulties as well as progress.
\end{abstract}

\newpage

\section{Project Purpose and Goals}
The purpose of the project is to build a set of open clock control software for the purpose of driving a letter display or BCD 
For the fall term there were many documents to write that will help to layout and define what we will be working on during the winter and spring terms next year.
Most of these documents will undergo revisions to make them better as we begin to better understand our project. 
That being said, the final product and goal are a fixed target, not including stretch goals, and are unlikely to change. 
\subsection{Problem Statement}
The problem statement was the first formal piece of writing to put the project onto paper. 
In this document we were working to create a definition and description of what our research and work will be based around for the rest of the year. 
By the end of the year, the problem described in this document is what we are seeking to solve.
For our project, the problem being sovled was making an affordable word clock that is friendly towards user modification.
\subsection{Requirements Document}
Once we had a better understanding of the specifics of the problem, a set of requirements were developed to guide the project.
In this document we were tasked with outlining the project so that both us and our client are clear on what the end product will be.
This document is important because it will serve as a list to create user stories and as a road map of sorts on when we expect to have certain feature completed by. 
\subsection{Technology Review}
After developing a set of requirements, it became clear that there were many aspects of the project would require additional information outside of what we started the project with.
In order to rectify these deficiencies, we selected nine important components from the project and took them through a technology review process.
For this document many groups would research possible solutions to their requirements and then choose which one was best suited for the task. 
But out project differed from most for this document, mainly because 3 of our main features were chosen for us by our project owner Victor.
The microcontroller, RTC, and LEDs that make up the main components of our clock were supplied by Victor.
So instead of finding which products to use for our clock, we wrote a comparison of alternative components to look into the quality of Victor's decisions. 
\subsection{Design Document}
The final technical document that was created this term was the Design Document.
In this we document we created a road map for development for the rest of the year. 
The details of this map may change, but the big pictures and final destination will stay the same.

\section{Current Stage of Project}
Currently in our project we are at a point where we are ready to start coding and designing features for the clock.
We have the main 3 components, the microcontroller, LEDs, and RTC, and now are starting the process of learning to connect them and communicate between the components.

\section{Problems and Solutions}
The main problems that we have run into right now is learning the hardware side of our project.
We have not researched the ways to connect all the hardware to the microcontroller.
Once we figure out how to connect everything writing all the code for the clock should come quickly since it is in C and not a very complex system.
Another problem we have had is time management on writing and planning out the documents outlined in the Project Purposes and Goals section. 
The main reason for our time management issues was that it usually took awhile for us to understand exactly what the assignment description was asking for.
We found the assignment descriptions a little vague at times and we had to contact our TA or a professor to fully understand assignments.
Meetings with our TA Jon Dodge were useful all term since we were able to ask questions and get a better understand of a document by asking him questions.

\section{Interesting Pieces of Code}
At this point of our project we have not written much code ourselves, but there are many hello world and demo programs that you can download for free on the MPLAB website.
These basic programs will help us to learn the basics of how to interact and control different parts of the board. 
Also the online IDE for the microcontroller is similar to Eclipse in layout and has a user friendly interface. 
To have version control for our code we will need to download the MPLAB IDE so that we can push and pull from git easier. 

\section{Retrospective of the Past 10 Weeks}
\begin{center}
\begin{tabular}{| p{0.3\linewidth} | p{0.3\linewidth} | p{0.3\linewidth} |}
\hline
Positives Column & 
Deltas Column & 
Actions Column \\
\hline
test text test text test text test text &
test text test text test text test text &
test text test text test text test text \\
\hline
\end{tabular}
\end{center}

\end{document}
