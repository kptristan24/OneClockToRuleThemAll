\documentclass[onecolumn, draftclsnofoot,10pt, compsoc]{IEEEtran}
\usepackage{graphicx}
\usepackage{url}
\usepackage{setspace}

\usepackage{geometry}
\geometry{textheight=9.5in, textwidth=7in}

% 1. Fill in these details
\def \CapstoneTeamName{     One Clock to Rule Them all}
\def \CapstoneTeamNumber{       57}
\def \GroupMemberOne{           Tasman Thenell}
\def \GroupMemberTwo{           Tristan Hari}
\def \GroupMemberThree{         Scott Metzsch}
\def \CapstoneProjectName{      One Clock to Rule Them all}
\def \CapstoneSponsorCompany{   Oregon State University}
\def \CapstoneSponsorPerson{        Dr. Victor Hsu}

% 2. Uncomment the appropriate line below so that the document type works

\def \DocType{		%Problem Statement
				%Requirements Document
				%Technology Review
				%Design Document
				Progress Report
				}



\newcommand{\NameSigPair}[1]{\par
\makebox[2.75in][r]{#1} \hfil   \makebox[3.25in]{\makebox[2.25in]{\hrulefill} \hfill        \makebox[.75in]{\hrulefill}}
\par\vspace{-12pt} \textit{\tiny\noindent
\makebox[2.75in]{} \hfil        \makebox[3.25in]{\makebox[2.25in][r]{Signature} \hfill  \makebox[.75in][r]{Date}}}}
% 3. If the document is not to be signed, uncomment the RENEWcommand below
\renewcommand{\NameSigPair}[1]{#1}

%%%%%%%%%%%%%%%%%%%%%%%%%%%%%%%%%%%%%%%
\begin{document}
\begin{titlepage}
    \pagenumbering{gobble}
    \begin{singlespace}
        %\includegraphics[height=4cm]{coe_v_spot1}
        \hfill
        % 4. If you have a logo, use this includegraphics command to put it on the coversheet.
        %\includegraphics[height=4cm]{CompanyLogo}
        \par\vspace{.2in}
        \centering
        \scshape{
            \huge CS Capstone \DocType \par
            {\large\today}\par
            \vspace{.5in}
            \textbf{\Huge\CapstoneProjectName}\par
            \vfill
            {\large Prepared for}\par
            \Huge \CapstoneSponsorCompany\par
            \vspace{5pt}
            {\Large\NameSigPair{\CapstoneSponsorPerson}\par}
            {\large Prepared by }\par
            Group\CapstoneTeamNumber\par
            % 5. comment out the line below this one if you do not wish to name your team
            %\CapstoneTeamName\par
            \vspace{5pt}
            {\Large
                \NameSigPair{\GroupMemberOne}\par
                \NameSigPair{\GroupMemberTwo}\par
                \NameSigPair{\GroupMemberThree}\par
            }
            \vspace{20pt}
        }
        \begin{abstract}
        % 6. Fill in your abstract
            The purpose of this document is to discuss the progress of the One Clock to Rule Them All project and what our team has done during the winter term.
Included is an explanation of each document updated during this period of time and a summary of the progress made during each week of the term.
In addition, this process is reflected in retrospective which discusses challenges and difficulties as well as progress.
        \end{abstract}
    \end{singlespace}
\end{titlepage}
\newpage
\pagenumbering{arabic}
\tableofcontents
% 7. uncomment this (if applicable). Consider adding a page break.
%\listoffigures
%\listoftables
\clearpage

% 8. now you write!
\section{Project Purpose and Goals}
The purpose of the project is to build a set of open clock control software for the purpose of driving a letter display or BCD
In addition to this, a clock housing will be developed with a clock face which satisfies the goals of legibility and flexibility.
Other goals include providing the final product at a price more reasonable than the thousand dollar asking price while stilling keeping a certain quality to the appearance.

\section{Current Stage of Project}
From our fall progress report to now we have made some major progress in getting closer to having our clock ready for expo in May.
Currently we have all the major components talking that would inhibit working on code.
We have the RTC communicating with the board and can set the time and display it in the serial monitor.
The LEDs are connected and correctly lighting up thanks to the libraries we found the connect them.
Getting the LEDs and RTC to connect and work was quite easy after moving to the new Arduino board since there are thousands of libraries that we can download and implement.
The change of the board has really increased our productivity and ability to move forward.
We also have our state machine format completely laid out and uploaded to git hub.

Looking forward we still have a bit to work on but most of the work is centered around designing and manufacturing the case for the clock.
We have successfully designed the word layout for the clock face itself, and have a preliminary face plate worked out in a 3d file.
The next step in that category will be to actually get the entire case laid out and iterate a couple possibilities there.
It shouldn't be too complex, but we are working through it now so that there are no surprises.
Designing some demos for the expo event are also on our to do list.
These demos will show control of color, brightness, different time display modes, and possibly games such as snake and pong.
Everything else is essentially lined up and ready to go for expo!

\section{Problems and Solutions}
We have had very few problems since our last progress update to be honest, so the ones included here will seem less impactful.
An important one being that we had a little trouble working out time

\section{Interesting Pieces of Code}
There are several awesome things that have pushed the project forward at an amazing rate.
First of all, we found an amazing library for interaction with the RTC from Dreamspark.
The library is compatible with our model RTC and contains extremely helpful functions such as RTC.Hour(), and RTC.Minute().
Having an RTC library this powerful is extremely useful and makes the idea of adding more flush features a reality.
We also found an awesome LED control library Adafruit.star.h, which has tonnes of documentation online and very powerful tools within it.
With it, we were very quickly able to generate a simple binary clock to test the library and the RTC reads in tandom.
It is very interesting.

As mentioned above, we managed to get essentially the entire skeleton of our project laid out in a state machine.
This is really a nice piece of progress as now we have defined the entire method in which we'll be displaying time, based off of the reads from the RTC.
It also leaves us sections to fill in for the later functionalities and button inputs.
This will include Menu states, game modes, settings mode, etc.

\section{Weekly Update Summaries}
\subsection{Week 1}
This week we mostly just got our bearings back on the status of the project and with our client, and prepared for the term.
We did not have much in terms of significance for the project progress.

\subsection{Week 2}
This week we got very back into fast pace development and we discovered something of major significance for the project.
After spending hours verifying a researching alternatives, we decided at this point that we needed to use a more friendly microcontroller.
We went to approach Victor with the KeyeStudio Uno, and see what he thinks.

\subsection{Week 3}
This week was excellent in terms of progress.
We got our new board, and we successfully got accurate time readings from the RTC reading to the serial monitor!
This is huge as is basically verifies the biggest part of the clocks functionality.

\subsection{Week 4}
This week we found several awesome RTC and LED libraries, that led to huge bounds in project progress.
We basically spent some time discovering functions and how we're going to use them.
We also got basic LED lights going.

\subsection{Week 5}
This week we met again with Victor and discussed the rest of the products that needed to be purchased for the completion of the product.
We settled buttons, powering, wiring, and LED details.

\subsection{Week 6}
This week we got our alpha prototype working.
It displays the time in BCD mode and shows all pieces properly interacting.
Next up will be to receive the buttons and figure out how those work.

\subsection{Week 7}
We did a lot of hardware planning and fabrication.
We fixed all the souldering on our original prototype and got the easy plug and play wires divided.
We also cut up all the LED strips we'll need for any prototyping, so that pretty nice.
Last but not least we got a draft of the .stl file needed to generate out 3d image face place for the clock, IE the letter grid.
This is just a preliminary grid so more iteration will be done.

\subsection{Week 8}
This week we worked on our word layout fairly extensively The trick is to attempt to match memes, functionality, and sizing for the grid.
We also got buttons interacting correctly.
This is very exciting because we were unsure if the buttons would have to interfere with our LED pins or not, but as it turns out we're absolutely good to go in that department.

\subsection{Week 9}
This week was a big one in terms of progress, we got a lot of the softwarre writing done.
Pretty much what remains is getting the scrolling text library figured out, software wise.

\subsection{Week 10}
This was a slow week for us, mostly due to school finally catching up in terms of speed.
We are looking into method of getting out clock face CnC'd and created so that we may properly bring a beta prototype to life.

\section{Teammate Evaluation}

\subsection{Tasman Thenell}
As usual, Tasman did an excellent job as a team member this term.
Always getting his portion done, and providing excellent dedication to ensuring the success of the end product.
Tas role I would say in the recent group is has been the software specialist, as he pretty much came up with the code structure and implemented it without too much input from the other members.
Tasmans Contributions to the group this term have been above and beyond the expectation and he should be recognized for them.
Having Tasman in the group has helped our group function in a well disciplined fashion, and I believe his presence has contributed to our great success so far.

\subsection{Scott Metszch}
Following right along, Scott has been a great team member as well this term.
Scott is very quick to come up with solutions and unique tools to approach our problems, which leads us to developing at a much faster rate.
If I were place Scott with a role, it would be technical specialist, as he seems to have experience in a lot of fields and it certainly has helped us thus far in the project.
Scotts level of contribution has been wonderfully satisfactory, he has done a lot of personal time to progress the project, and that is great.
Having Scott on the team really coheses our teammates personalities very well and provides a lot of development insight.

\subsection{Me}
What can I say, I am wonderful.
All jokes aside I consider myself fairly decent at keeping my work done, as well as coordinating out individual pieces to make a cohesive project.
I consider my role to be the Coordinator, Communicator, and Solution finder.
I tend to be quick with words and speaking with our client, as well as trying to nail down meeting times and work times.
A lot of what I've contributed to the project aside from simple individual labor like preliminary 3d generation and some code, is found in the research i have done to keep our group moving along the right track.
I tend to go out and find the resources and whatnot we need to keep the project moving.
I can improve in my ability to properly CC my teammates on email chains, as well as personal time for project contribution.
Overall I feel like my contributions are useful to the team and certainly aid the completion of the project.

\newpage
\section{Retrospective of the Past 10 Weeks}
\vspace{2mm}
\begin{center}
\begin{tabular}{| p{0.3\linewidth} | p{0.3\linewidth} | p{0.3\linewidth} |}
\hline
Positives Column &
Deltas Column &
Actions Column \\
\hline
Changed board to KeyeStudio Uno board &
Get RTC connected and reading &
Get LED connected and outputting \\
\hline
 &
 &
Begin work on project starting with hardware related steps.  \\
\hline
Changed board to KeyeStudio Uno board &
 &
New Board and libraries and coding environment \\
\hline
Received RTC from Victor &
Need to learn connectivity and find power source &
Research how to properly connect microcontroller and find what type of battery the RTC uses \\
\hline
 &
 &
Develop Case and letter grid \\
\hline
Finished initial draft of product requirements and design. &
&
Design pin assignment and \\
\hline
Develop tests &
Finished initial draft of product requirements and design. &
\\
\hline


\end{tabular}
\end{center}

\end{document}
