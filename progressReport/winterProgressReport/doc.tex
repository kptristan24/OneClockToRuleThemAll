\documentclass[onecolumn, draftclsnofoot,10pt, compsoc]{IEEEtran}
\usepackage{graphicx}
\usepackage{url}
\usepackage{setspace}

\usepackage{geometry}
\geometry{textheight=9.5in, textwidth=7in}

% 1. Fill in these details
\def \CapstoneTeamName{		One Clock to Rule Them all}
\def \CapstoneTeamNumber{		57}
\def \GroupMemberOne{			Tasman Thenell}
\def \GroupMemberTwo{			Tristan Hari}
\def \GroupMemberThree{			Scott Metzsch}
\def \CapstoneProjectName{		One Clock to Rule Them all}
\def \CapstoneSponsorCompany{	Oregon State University}
\def \CapstoneSponsorPerson{		Dr. Victor Hsu}

% 2. Uncomment the appropriate line below so that the document type works
\def \DocType{		%Problem Statement
				%Requirements Document
				%Technology Review
				%Design Document
				Progress Report
				}

\newcommand{\NameSigPair}[1]{\par
\makebox[2.75in][r]{#1} \hfil 	\makebox[3.25in]{\makebox[2.25in]{\hrulefill} \hfill		\makebox[.75in]{\hrulefill}}
\par\vspace{-12pt} \textit{\tiny\noindent
\makebox[2.75in]{} \hfil		\makebox[3.25in]{\makebox[2.25in][r]{Signature} \hfill	\makebox[.75in][r]{Date}}}}
% 3. If the document is not to be signed, uncomment the RENEWcommand below
\renewcommand{\NameSigPair}[1]{#1}

%%%%%%%%%%%%%%%%%%%%%%%%%%%%%%%%%%%%%%%
\begin{document}
\begin{titlepage}
    \pagenumbering{gobble}
    \begin{singlespace}
    	%\includegraphics[height=4cm]{coe_v_spot1}
        \hfill
        % 4. If you have a logo, use this includegraphics command to put it on the coversheet.
        %\includegraphics[height=4cm]{CompanyLogo}
        \par\vspace{.2in}
        \centering
        \scshape{
            \huge CS Capstone \DocType \par
            {\large\today}\par
            \vspace{.5in}
            \textbf{\Huge\CapstoneProjectName}\par
            \vfill
            {\large Prepared for}\par
            \Huge \CapstoneSponsorCompany\par
            \vspace{5pt}
            {\Large\NameSigPair{\CapstoneSponsorPerson}\par}
            {\large Prepared by }\par
            Group\CapstoneTeamNumber\par
            % 5. comment out the line below this one if you do not wish to name your team
            %\CapstoneTeamName\par
            \vspace{5pt}
            {\Large
                \NameSigPair{\GroupMemberOne}\par
                \NameSigPair{\GroupMemberTwo}\par
                \NameSigPair{\GroupMemberThree}\par
            }
            \vspace{20pt}
        }
        \begin{abstract}
        % 6. Fill in your abstract
        	The purpose of this document is to discuss the progress of the One Clock to Rule Them All project and what our team has done during the fall term.
Included is an explanation of each document written during the design period and a summary of the progress made during each week of the term.
In addition, this process is reflected in retrospective which discusses challenges and difficulties as well as progress.
        \end{abstract}
    \end{singlespace}
\end{titlepage}
\newpage
\pagenumbering{arabic}
\tableofcontents
% 7. uncomment this (if applicable). Consider adding a page break.
%\listoffigures
%\listoftables
\clearpage

% 8. now you write!
\section{Project Purpose and Goals}
The purpose of the project is to build a set of open clock control software for the purpose of driving a letter display or BCD
For the fall term there were many documents to write that will help to layout and define what we will be working on during the winter and spring terms next year.
Most of these documents will undergo revisions to make them better as we begin to better understand our project.
That being said, the final product and goal are a fixed target, not including stretch goals, and are unlikely to change.
\subsection{Problem Statement}
The problem statement was the first formal piece of writing to put the project onto paper.
In this document we were working to create a definition and description of what our research and work will be based around for the rest of the year.
By the end of the year, the problem described in this document is what we are seeking to solve.
For our project, the problem being sovled was making an affordable word clock that is friendly towards user modification.
\subsection{Requirements Document}
Once we had a better understanding of the specifics of the problem, a set of requirements were developed to guide the project.
In this document we were tasked with outlining the project so that both us and our client are clear on what the end product will be.
This document is important because it will serve as a list to create user stories and as a road map of sorts on when we expect to have certain feature completed by.
\subsection{Technology Review}
After developing a set of requirements, it became clear that there were many aspects of the project would require additional information outside of what we started the project with.
In order to rectify these deficiencies, we selected nine important components from the project and took them through a technology review process.
For this document many groups would research possible solutions to their requirements and then choose which one was best suited for the task.
But out project differed from most for this document, mainly because 3 of our main features were chosen for us by our project owner Victor.
The microcontroller, RTC, and LEDs that make up the main components of our clock were supplied by Victor.
So instead of finding which products to use for our clock, we wrote a comparison of alternative components to look into the quality of Victor's decisions.
\subsection{Design Document}
The final technical document that was created this term was the Design Document.
In this we document we created a road map for development for the rest of the year.
The details of this map may change, but the big pictures and final destination will stay the same.

\section{Current Stage of Project}
The project has reached the stage where we know where we are going but we need to address some hurdles in hardware development before we can progress furthor.
While there are many exiting ideas and features we would like to implement, they all sit behind a hardware barrier to entry that we need to address.
We have the main 3 components, the microcontroller, LEDs, and RTC, and now are starting the process of learning to connect them and communicate between the components.
In addition to this, we have several different examples of how other developers have gone about designing software for powering word clocks.
We plan on using these examples as resources and references although none of them cover any of the advanced functionality that is planned.
\section{Problems and Solutions}
The main problems that we have run into right now is learning the hardware side of our project.
We have not researched the ways to connect all the hardware to the microcontroller, but have started to figure this out from experimentation and documentation.
Once we figure out how to connect everything, writing all the code for the clock should come quickly since it is in C and everybody on the team has relevant coding and embedded experience.
Another problem we have had is time management on writing and planning out the documents outlined in the Project Purposes and Goals section.
The main reason for our time management issues was that it usually took awhile for us to understand exactly what the assignment description was asking for.
We found the assignment descriptions a little vague at times and we had to contact our TA or a professor to fully understand assignments.
Meetings with our TA Jon Dodge were useful all term since we were able to ask questions and get a better understand of a document by asking him questions.

\section{Interesting Pieces of Code}
There are several awesome things that have pushed the project forward at an amazing rate.
First of all, we found an amazing library for interaction with the RTC from Dreamspark.
The library is compatible with our model RTC and contains extremely helpful functions such as RTC.Hour(), and RTC.Minute().
Having an RTC library this powerful is extremely useful and makes the idea of adding more flush features a reality.
We also found an awesome LED control library Adafruit.star.h, which has tonnes of documentation online and very powerful tools within it.
With it, we were very quickly able to generate a simple binary clock to test the library and the RTC reads in tandom.
It is very interesting.

\section{Weekly Update Summaries}
\subsection{Week 1}
This week we mostly just got our bearings back on the status of the project and with our client, and prepared for the term.
We did not have much in terms of significance for the project progress.

\subsection{Week 2}
This week we got very back into fast pace development and we discovered something of major significance for the project.
After spending hours verifying a researching alternatives, we decided at this point that we needed to use a more friendly microcontroller.
We went to approach Victor with the KeyeStudio Uno, and see what he thinks.

\subsection{Week 3}
This week was excellent in terms of progress.
We got our new board, and we successfully got accurate time readings from the RTC reading to the serial monitor!
This is huge as is basically verifies the biggest part of the clocks functionality.

\subsection{Week 4}
This week we found several awesome RTC and LED libraries, that led to huge bounds in project progress.
We basically spent some time discovering functions and how we're going to use them.
We also got basic LED lights going.

\subsection{Week 5}
This week we met again with Victor and discussed the rest of the products that needed to be purchased for the completion of the product.
We settled buttons, powering, wiring, and LED details.

\newpage
\section{Retrospective of the Past 10 Weeks}
\vspace{2mm}
\begin{center}
\begin{tabular}{| p{0.3\linewidth} | p{0.3\linewidth} | p{0.3\linewidth} |}
\hline
Positives Column &
Deltas Column &
Actions Column \\
\hline
Received first set of LEDs from Victor &
Change the spacing of the LEDs to accommodate letter size and position &
Cut LEDs on specified lines and add extensions to increase the distance between LEDs \\
\hline
 &
 &
Learn how to transfer data to LEDs that will control colors and which LEDs are currently lit \\
\hline
Received second set of LEDs from Victor, this time with different LED spacing &
 &
Check if the default spacing of the new LEDs will work for the letter grid layout on the clock face \\
\hline
Received RTC from Victor &
Need to learn connectivity and find power source of RTC &
Research how to properly connect the RTC to the microcontroller and find what type of battery the RTC uses \\
\hline
 &
 &
Do short term test to find the time loss of the RTC and if it is within requirements \\
\hline
Finished initial draft of droduct requirements and design. &
&
Begin work on project starting with hardware related steps. \\
\hline
Discovered board IDE features. &
Changed how we were planning to approach software development so a more simplified workflow. &
Figure out how to mesh this with a version control system. \\
\hline


\end{tabular}
\end{center}

\end{document}
