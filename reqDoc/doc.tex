%\documentclass[10pt,draftclsnofoot,onecolumn]{IEEEtran}
\documentclass[10pt,letterpaper,onecolumn]{article}
\newcommand{\namesigdate}[2][7cm]{%
\begin{minipage}{#1}
    #2 \vspace{1.0cm}\hrule\smallskip
    \small \textit{Signature}
    \vspace{1.0cm}\hrule\smallskip
    \small \textit{Date}
\end{minipage}
}
%\newcommand{\subparagraph}{} necessary for using ieeetran

\usepackage{enumerate}
\usepackage[shortlabels]{./enumitem}
\usepackage{graphicx}
\usepackage{textcomp}
\usepackage{titlesec}
\usepackage[utf8]{inputenc}
\usepackage[T1]{fontenc}
\usepackage{lmodern}
\usepackage{datetime}
\newdate{date}{11}{05}{2016}
\date{\displaydate{date}}

\graphicspath{ {} }

\setcounter{secnumdepth}{5}
\renewcommand\thesection{\arabic{section}}
\renewcommand\thesubsection{\arabic{section}.\arabic{subsection}}
\renewcommand\thesubsubsection{\arabic{section}.\arabic{subsection}.\arabic{subsubsection}}
\renewcommand\theparagraph{\arabic{section}.\arabic{subsection}.\arabic{subsubsection}.\arabic{paragraph}}
\setlength\voffset{-1in}
\setlength\hoffset{-1in}
\setlength\topmargin{0.5in}
\setlength\oddsidemargin{1in}
\setlength\evensidemargin{1in}
\setlength\textheight{8.278in}
\setlength\textwidth{6.5in}
\setlength\footskip{0.561in}
\setlength\headheight{0.5in}
\setlength\headsep{0.461in}

\renewcommand{\sfdefault}{phv}

\titleformat*{\section}{\Large\bfseries\sffamily}
\titleformat*{\subsection}{\large\bfseries\sffamily}
\titleformat*{\subsubsection}{\large\bfseries\sffamily}

\setlength{\skip\footins}{0.0469in}
\renewcommand\footnoterule{\vspace*{-0.0071in}\setlength\leftskip{0pt}\setlength\rightskip{0pt plus 1fil}\noindent\textcolor{black}{\rule{0.25\columnwidth}{0.0071in}}\vspace*{0.0398in}}



\begin{document}

\begin{titlepage}
    \begin{flushleft}
        \vspace*{1cm}
        \begin{Huge}
          \sffamily {\textbf{OneClockToRuleThemAll Requirements and Specifications}}
        \end{Huge}
        \vspace{4 cm}

        Sponsor \\

        \begin{large}
          \sffamily { \textbf{Dr. Victor Hsu}} \\
          \sffamily { \textbf{Oregon State University}} \\
          \sffamily { \textbf{Department of Biochemistry and Biophysics}} \\
        \end{large}

        \vspace{1.5cm}

        Authors\\
        \begin{large}
          \sffamily { \textbf{Tasman Thenell}} \\
          \sffamily { \textbf{Tristan Hari}} \\
          \sffamily { \textbf{Scott Metzsch}} \\
        \end{large}
        \vspace{.5cm}
        Submission Date \\
        \begin{large}
          \sffamily { \textbf{\protect\displaydate{date}}} \\
        \end{large}
        \vspace{2 cm}
        \begin{sffamily}
        \textbf{Abstract: }
        This document outlines the requirements and specifications for our senior capstone project, "One Clock To Rule Them All."
        In the sections below we will give a better understanding of the overall idea of the project, as
        well as a more specific break down of what exactly will be done, how the product will perform,
        and what one could come to expect from a user experience of using the end product. We will also
        cover detailed technical information and ranges of operation for the device. Overall the clock in
        itself is a very simple device and fairly straightforward, however we will discuss the details here.

        \end{sffamily}
        \vfill

    \end{flushleft}
\end{titlepage}

\pagenumbering{roman}
\newpage
\tableofcontents
\newpage

\pagenumbering{arabic}

\section{Introduction}
\subsection{Purpose}
Historically, the clock face has had one of two display modes, analog or digital.
These formats have provided a solid mix between function and fashion, but there are many
other possible ways of visually representing the concept of time. One creative idea worth
exploring is timekeeping through words and letters. Although some examples of this already
exist in the high end fashion market, such as QWLock or others, they are well outside the
price range of an average consumer. The purpose of our project is to provide a similar
product with much cheaper production.

\subsection{Scope}
The name of our product is tentatively “One Clock To Rule Them All”, or “The One Clock.”
The One Clock’s function will be, simply, to tell time. However the functionality of it is
defined as being able to creatively tell time through a grid of apparently random letters.
Time will be displayed in forms such as “It is half past twelve” or “it is one fifty.” The
user will be able to set the starting time using a button component on the clock. Something
that may be assumed as a regular clock function that our clock will not do, is that this clock
 is not centered around being an alarm/scheduler device. We do not intend to have any sound
 components and thusly any alarm system will be mostly moot.

\subsection{Definitions}
\begin{enumerate}[a)]
  \item Time - The abstract concept that our machine is keeping track of and displaying to the user
  \item Real Time Clock - A real time clock module is the hardware component that tracks the passage
  of time regardless of the state of the rest of the system.
  \item RTC - Acronym for Real Time Clock module.
  \item Microcontroller - The brains of the clock. This module interfaces between the hardware
  components, the RTC, LEDs and buttons, and processes the logic for all clock functionality.
  This includes tasks like updating the display and processing user input from the buttons.
  \item Programmable LED - A Light Emitting Diode that has variable levels of brightness and
  color output. Each module is individually programmable in terms of brightness and color.
  \item LED - Acronym for Light Emitting Diode. Within this project, the term LED will always
  be used to refer to a programmable light emitting diode.
  \item Display - The LED lit letters of the clock face in a grid arrangement are collectively
  called the display. This term will be used to refer to the visual
  \item Software Library - The set of control software running on the microcontroller which drives all display output.
  \item User Interface - A set of buttons by which the user controls and interacts with the clock.
  \item BCD - Acronym for Binary Coded Decimal.
  \item Binary Coded Decimal - A method for encoding decimal numbers and information via binary.
  \item USB - Acronym for Universal Serial Bus.
  \item Universal Serial Bus - A communications medium by which the device is programmed and  powered.
\end{enumerate}

\subsection{Overview}
The following sections of this requirements document describe product specifications as follows.
Section 2 explains the purpose, function, and form of the product in broad terms on a less
technical level. Building on that foundation, Section 3 breaks the functional requirements
describe in Section 2 into technical requirements and specifications.
	Both of the following sections are intended for conveying the entire functional
scope of the Word Clock project but at different levels. Section 2 provides an overview of the
project while Section 3 pertains to exact requirements which requires a more technical working
knowledge of hardware and software design.

\section{Overall Description}
\subsection{Product Perspective}
The clock we are making will be completely self contained with a battery
or wall plug being used to power the clock. There will be a microcontroller that is used to
control the LEDs on the display and interface with the RTC in order to provide accurate
timekeeping capabilities.

On the side of the clock there will be 4 buttons that will be used to set the time on the
clock and change features. Each button provides a specific function depending on the software
context. An example of this system when a basic menu is being manipulated would have two
buttons providing up and down scrolling, one button to select an option or submenu, and the
last button serving as a back or exit button.

The microcontroller will be the central piece of hardware in our clock and have the software
for running it embedded into it. The RTC will be connected to the board through 6 ports and
will be used to check the time periodically to verify that the time is still correct. The LEDs
will also be connected to the board and the microcontroller will send data to the LEDs that
tell them which LEDs to light up. Lastly the 4 buttons will be connected to the
microcontroller to set time and access additional features added to the clock.

\subsection{Product Functions}
Expected functionality related to standard clock features is known at this time
but other possible features have yet to be determined. A vague reference to these
future items is listed as the last function but no requirements for this item
is included in section 3.
\begin{enumerate}[a)]
  \item The main function of the clock will be to tell time through words displayed on the face of the
clock. Examples of the output format include “it is a quarter past twelve”, “it is a quarter to
five.”
  \item The clock shall provide a way for the user to set and edit alarms. While
  making an audible alarm isn't planned, user programmible alarms indicated by
  the display flashing are expected functionality.
  \item Additional functionality includes the possibility of adding extra features to the clock such as
color changes to tell the time of day, differences in brightness, and other features whose
development is purely contingent on time constraints.
\end{enumerate}

\subsection{User Characteristics}
Users of this product are expected to fall into two categories. While that might
seem to over generalize things, users break down cleanly into two categories.
The first is normal users who are just using this product as a clock. The second
type of user is the poweruser who might want to tinker with the software and
hardware of the clock in order to understand or expand upon it.
\begin{enumerate}[a)]
  \item A normal user is expected to be able to read, tell time and use a manual.
  \item Developers are expected to have the skills of a user as well as knowing the
capabilities of the hardware interfaces and the microcontroller in addition to
any skills necessary for software development.
\end{enumerate}

\subsection{Constraints}
The project needs to be meet a small set of physical and architectural constraints. These
constraints center around the units ability to act like a traditional clock in a reasonable
amount of space.
\begin{enumerate}[a)]
  \item The product must be accurate over the course of the year.
  \item The system needs to be compact enough to fit in a normal wall clock.
  \item Power shall be supplied via a conventional micro USB feed.
\end{enumerate}

\section{Specific Requirements}
\subsection{External Interface}
A set of 4 buttons provide the only interface to the software. These buttons
allow access to interacting with any available software menus. This is the only
point of contact between the user and the software.

\subsection{Functions}
The following items break down the general concepts of acting like a clock into individual
conponents of functionality. Together these functions are the software side of
how this product will be able to act like a clock.
\begin{enumerate}[a)]
  \item The system shall offer the user the ability to set the time via a
  physical interface.
  \item
  The system shall allow for setting, modification, and viewing of alarms.
  \item
  The system shall provide a visual indicator of an alarm going off but will not
  provide any auditory notification.
  \item
  The system shall recover from any series of inputs via the four button
  interface. Any input from other sources is considered unsupported and may result
  in undefined behavior.
  \item
  The system shall provide a menu option for switching output mode into BCD as
  well as a method for returning from BCD to the normal word output mode.
  \item
  The system shall update the time displayed every minute but the time in words
  shall be updated only every five minutes.
  \item
  The system shall provide a visual indicator for single minute increments of
  time.
\end{enumerate}

\subsection{Performance Requirements}
\begin{enumerate}[a)]
  \item The clock shall update the time being displayed within a second of the
  time in minutes changing.
  \item The clock shall not lose more than .432 seconds of accuracy per day.
  \item The system shall maintain this accuracy regardless of the presence of
  power for at least nine years, the lifespan of the RTC internal battery.
\end{enumerate}

\subsection{Physical Requirements}

\begin{enumerate}[a)]
  \item The system components, not including the clock housing, shall weigh no
  more than 5 pounds.
  \item The display shall be legible at a range of up to 10 feet for users with
  typical visual acuity, roughly equivalent to a 20/20 rating.
  \item The clock shall operate with the accuracy described in section 3.3.2 and
  3.3.3 within the temperature range of 0 degrees to 85 degrees fahrenheit.
  \item The grid of letters used to display the time will be no smaller than 10
  by 10 and no larger than 25 by 25 letters.
  \item The system shall be powered by a micro USB port.
  \item The system shall have four tactile buttons.
\end{enumerate}

\newpage

\noindent \namesigdate{Victor Hsu} \hfill \namesigdate{Tristan Hari} \par
\vspace{2cm}
\noindent \namesigdate{Tasman Thenell} \hfill \namesigdate{Scott Metzsch}
\end{document}
