\documentclass[10pt,draftclsnofoot,onecolumn]{IEEEtran}
\newcommand{\namesigdate}[2][7cm]{%
\begin{minipage}{#1}
    #2 \vspace{1.0cm}\hrule\smallskip
    \small \textit{Signature}
    \vspace{1.0cm}\hrule\smallskip
    \small \textit{Date}
\end{minipage}
}
\usepackage{graphicx}
\usepackage{textcomp}
\graphicspath{ {} }

\renewcommand\thesection{\arabic{section}}

\makeatletter
\renewcommand\section{\@startsection {section}{1}{\z@}%
                                   {-3.5ex \@plus -1ex \@minus -.2ex}%
                                   {2.3ex \@plus.2ex}%
                                   {\normalfont\LARGE\bfseries}}% from \Large
\makeatother


\begin{document}
\pagenumbering{gobble}
\title{One Clock To Rule Them All Requirements Doc}
\author{Tristan Hari, Tasman Thenell, and Scott Metzsch}
\maketitle
\begin{abstract}
This will be our requirements document for our senior capstone project, "One Clock To Rule Them All."
In the sections below we will give a better understanding of the overall idea of the project, as
well as a more specific break down of what exactly will be done, how the product will perform,
and what one could come to expect from a user experience of using the end product. We will also
cover detailed technical information and ranges of operation for the device. Overall the clock in
itself is a very simple device and fairly straightforward, however we will discuss the details here.
\end{abstract}
\IEEEpeerreviewmaketitle

\newpage
\tableofcontents
\newpage

\pagenumbering{arabic}

\title{Intro}
\section{Purpose}
Historically, the clock face has had one of two display modes, analog or digital.
These formats have provided a solid mix between function and fashion, but there are many
other possible ways of visually representing the concept of time. One creative idea worth
exploring is timekeeping through words and letters. Although some examples of this already
exist in the high end fashion market, such as QWLock or others, they are well outside the
price range of an average consumer. The purpose of our project is to provide a similar
product with much cheaper production.

\section{Scope}
The name of our product is tentatively “One Clock To Rule Them All”, or “The One Clock.”
The One Clock’s function will be, simply, to tell time. However the functionality of it is
defined as being able to creatively tell time through a grid of apparently random letters.
Time will be displayed in forms such as “It is half past twelve” or “it is one fifty.” The
user will be able to set the starting time using a button component on the clock. Something
that may be assumed as a regular clock function that our clock will not do, is that this clock
 is not centered around being an alarm/scheduler device. We do not intend to have any sound
 components and thusly any alarm system will be mostly moot.

\section{Definitions}
\begin{itemize}
  \item Time - The abstract concept that our machine is keeping track of and displaying to the user
  \item Real Time Clock - A real time clock module is the hardware component that tracks the passage
  of time regardless of the state of the rest of the system.
  \item RTC - Acronym for Real Time Clock module.
  \item Microcontroller - The brains of the clock. This module interfaces between the hardware
  components, the RTC, LEDs and buttons, and processes the logic for all clock functionality.
  This includes tasks like updating the display and processing user input from the buttons.
  \item Programmable LED - A Light Emitting Diode that has variable levels of brightness and
  color output. Each module is individually programmable in terms of brightness and color.
  \item LED - Acronym for Light Emitting Diode. Within this project, the term LED will always
  be used to refer to a programmable light emitting diode.
  \item Display - The LED lit letters of the clock face in a grid arrangement are collectively
  called the display. This term will be used to refer to the visual
  \item Software Library - The set of control software running on the microcontroller which drives all display output.
  \item User Interface - A set of buttons by which the user controls and interacts with the clock.
  \item BCD - Acronym for Binary Coded Decimal.
  \item Binary Coded Decimal - A method for encoding decimal numbers and information via binary.
  \item USB - Acronym for Universal Serial Bus.
  \item Universal Serial Bus - A communications medium by which the device is programmed and  powered.
\end{itemize}

\section{Overview}
The following sections of this requirements document describe product specifications as follows.
Section 2 explains the purpose, function, and form of the product in broad terms on a less
technical level. Building on that foundation, Section 3 breaks the functional requirements
describe in Section 2 into technical requirements and specifications.
	Both of the following sections are intended for conveying the entire functional
scope of the Word Clock project but at different levels. Section 2 provides an overview of the
project while Section 3 pertains to exact requirements which requires a more technical working
knowledge of hardware and software design.

\newpage
\title{Overall Description}
\subsection{Product Perspective}
Product perspective -The clock we are making will be completely self contained with a battery
or wall plug being used to power the clock. There will be a microcontroller that is used to
control the LEDs on the display and interface with the RTC in order to provide accurate
timekeeping capabilities.

On the side of the clock there will be 4 buttons that will be used to set the time on the
clock and change features. Each button provides a specific function depending on the software
context. An example of this system when a basic menu is being manipulated would have two
buttons providing up and down scrolling, one button to select an option or submenu, and the
last button serving as a back or exit button.

The microcontroller will be the central piece of hardware in our clock and have the software
for running it embedded into it. The RTC will be connected to the board through 6 ports and
will be used to check the time periodically to verify that the time is still correct. The LEDs
will also be connected to the board and the microcontroller will send data to the LEDs that
tell them which LEDs to light up. Lastly the 4 buttons will be connected to the
microcontroller to set time and access additional features added to the clock.

\sub section{Product Functions}
The main function of the clock will be to tell time through words displayed on the face of the
clock. Examples of the output format include “it is a quarter past twelve”, “it is a quarter to
five.”

Additional functionality includes the possibility of adding extra features to the clock such as
color changes to tell the time of day, differences in brightness, and other features whose
development is purely contingent on time constraints.

\subsection{User Characteristics}
\begin{itemize}
  \item A normal user is expected to be able to read, tell time and use a manual.
  \item Developers are expected to have the skills of a user as well as knowing the
  capabilities of the hardware interfaces and the microcontroller in addition to any skills
  necessary for software development.
\end{itemize}

\subsection{Constraints}
The project needs to be meet a small set of physical and architectural constraints. These
constraints center around the units ability to act like a traditional clock in a reasonable
amount of space.
\begin{itemize}
	\item The product must be accurate over the course of the year.
	\item The system needs to be compact enough to fit in a normal wall clock.
	\item Power shall be supplied via a conventional micro USB feed.
\end{itemize}

\newpage
\title{Specific Requirements}
\subsection{Breakdown}
The system shall tell time within an accuracy of .432 seconds per day. The system will allow
the user to set the time within five minutes of the exact desired time. The user will use the
buttons on the clock to change time effectively. The clock shall be able to retain it’s time
state even if it is without external power for sometime. The system shall display time in a
standard twelve hour format using words. The output of the time will be understood by the user
in a reasonable manner. The amount of supported users is only limited by the scope of said
users vision or the visibility of the device to the users line of sight. Direct operation is
not required to view the time. The product will operate as desired within the temperature
range of -40 to 85 degrees fahrenheit.

\newpage

\noindent \namesigdate{Victor Hsu} \hfill \namesigdate{Tristan Hari} \par
\vspace{2cm}
\noindent \namesigdate{Tasman Thenell} \hfill \namesigdate{Scott Metzsch}
\end{document}
